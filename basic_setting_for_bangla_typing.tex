\documentclass{book}
\usepackage[banglamainfont=Kalpurush, banglattfont=Siyam Rupali]{latexbangla}
\setdefaultlanguage[numerals=Bengali, changecounternumbering=true]{bengali}
\usepackage[paperwidth=5.5in, paperheight=8.5in]{geometry}

\begin{document}
\tableofcontents
\chapter{কংগ্রুয়েন্স}
\section{কংগ্রুয়েন্স}
\begin{theorem}[ফার্মার লিটল থিওরেম]
$p$ একটি মৌলিক সংখ্যা এবং $n$ একটি স্বাভাবিক সংখ্যা হলে,
\[n^p\equiv n\pmod{p}\]
\end{theorem}
লক্ষ্য করুন, এখন পর্যন্ত টেক্সট প্রদর্শনের জন্য \textbf{কালপুরুষ} ফ্রন্ট ব্যবহার করা হয়েছে।
গাজীপুর সিটির নির্বাচনে ৪৮০ কেন্দ্রের মধ্যে ৪০টি কেন্দ্রের বেসরকারি ফল ঘোষণা হয়েছে। আজ বৃহস্পতিবার রাত নয়টার দিকে এ ফল জানানো হয়। এতে দেখা যায়, নৌকা মার্কার প্রার্থী আজমত উল্লা খান পেয়েছেন ২০ হাজার ৮৪৮ ভোট। আর টেবিল ঘড়ি প্রতীকে জায়েদা খাতুন পেয়েছেন ২১ হাজার ৩৭ ভোট। 

নির্বাচন কমিশনের তথ্যানুযায়ী, গাজীপুর সিটিতে মোট ভোটার ১১ লাখ ৭৯ হাজার ৪৭৬ জন। তাঁদের মধ্যে ৫ লাখ ৯২ হাজার ৭৬২ জন পুরুষ, ৫ লাখ ৮৬ হাজার ৬৯৬ জন নারী ও ১৮ জন হিজড়া। এই সিটিতে ৫৭টি সাধারণ ও ১৯টি সংরক্ষিত ওয়ার্ড আছে। মোট ভোটকেন্দ্র ৪৮০টি, মোট ভোটকক্ষ ৩ হাজার ৪৯৭টি।

\section{গাজীপুর}
নির্বাচনের মেয়র প্রার্থীরা হলেন নৌকা প্রতীকে আওয়ামী লীগের প্রার্থী আজমত উল্লা খান, টেবিলঘড়ি প্রতীকে জায়েদা খাতুন (সাবেক মেয়র জাহাঙ্গীর আলমের মা), লাঙ্গল প্রতীকে জাতীয় পার্টির প্রার্থী এম এম নিয়াজ উদ্দিন, হাতপাখা প্রতীকে ইসলামী আন্দোলন বাংলাদেশের গাজী আতাউর রহমান, গোলাপ ফুল প্রতীকে জাকের পার্টির মো. রাজু আহাম্মেদ, মাছ প্রতীকে গণফ্রন্টের প্রার্থী আতিকুল ইসলাম। এ ছাড়া স্বতন্ত্র থেকে মেয়র পদে ঘোড়া প্রতীকে মো. হারুন-অর-রশীদ ও হাতি প্রতীকে সরকার শাহনূর ইসলাম প্রতিদ্বন্দ্বিতা করেছেন।

গাজীপুর সিটির নির্বাচনে ৪৮০ কেন্দ্রের মধ্যে ৪০টি কেন্দ্রের বেসরকারি ফল ঘোষণা হয়েছে। আজ বৃহস্পতিবার রাত নয়টার দিকে এ ফল জানানো হয়। এতে দেখা যায়, নৌকা মার্কার প্রার্থী আজমত উল্লা খান পেয়েছেন ২০ হাজার ৮৪৮ ভোট। আর টেবিল ঘড়ি প্রতীকে জায়েদা খাতুন পেয়েছেন ২১ হাজার ৩৭ ভোট। 

নির্বাচন কমিশনের তথ্যানুযায়ী, গাজীপুর সিটিতে মোট ভোটার ১১ লাখ ৭৯ হাজার ৪৭৬ জন। তাঁদের মধ্যে ৫ লাখ ৯২ হাজার ৭৬২ জন পুরুষ, ৫ লাখ ৮৬ হাজার ৬৯৬ জন নারী ও ১৮ জন হিজড়া। এই সিটিতে ৫৭টি সাধারণ ও ১৯টি সংরক্ষিত ওয়ার্ড আছে। মোট ভোটকেন্দ্র ৪৮০টি, মোট ভোটকক্ষ ৩ হাজার ৪৯৭টি।

\section{গাজীপুর}
নির্বাচনের মেয়র প্রার্থীরা হলেন নৌকা প্রতীকে আওয়ামী লীগের প্রার্থী আজমত উল্লা খান, টেবিলঘড়ি প্রতীকে জায়েদা খাতুন (সাবেক মেয়র জাহাঙ্গীর আলমের মা), লাঙ্গল প্রতীকে জাতীয় পার্টির প্রার্থী এম এম নিয়াজ উদ্দিন, হাতপাখা প্রতীকে ইসলামী আন্দোলন বাংলাদেশের গাজী আতাউর রহমান, গোলাপ ফুল প্রতীকে জাকের পার্টির মো. রাজু আহাম্মেদ, মাছ প্রতীকে গণফ্রন্টের প্রার্থী আতিকুল ইসলাম। এ ছাড়া স্বতন্ত্র থেকে মেয়র পদে ঘোড়া প্রতীকে মো. হারুন-অর-রশীদ ও হাতি প্রতীকে সরকার শাহনূর ইসলাম প্রতিদ্বন্দ্বিতা করেছেন।

গাজীপুর সিটির নির্বাচনে ৪৮০ কেন্দ্রের মধ্যে ৪০টি কেন্দ্রের বেসরকারি ফল ঘোষণা হয়েছে। আজ বৃহস্পতিবার রাত নয়টার দিকে এ ফল জানানো হয়। এতে দেখা যায়, নৌকা মার্কার প্রার্থী আজমত উল্লা খান পেয়েছেন ২০ হাজার ৮৪৮ ভোট। আর টেবিল ঘড়ি প্রতীকে জায়েদা খাতুন পেয়েছেন ২১ হাজার ৩৭ ভোট। 

নির্বাচন কমিশনের তথ্যানুযায়ী, গাজীপুর সিটিতে মোট ভোটার ১১ লাখ ৭৯ হাজার ৪৭৬ জন। তাঁদের মধ্যে ৫ লাখ ৯২ হাজার ৭৬২ জন পুরুষ, ৫ লাখ ৮৬ হাজার ৬৯৬ জন নারী ও ১৮ জন হিজড়া। এই সিটিতে ৫৭টি সাধারণ ও ১৯টি সংরক্ষিত ওয়ার্ড আছে। মোট ভোটকেন্দ্র ৪৮০টি, মোট ভোটকক্ষ ৩ হাজার ৪৯৭টি।

\section{গাজীপুর}
নির্বাচনের মেয়র প্রার্থীরা হলেন নৌকা প্রতীকে আওয়ামী লীগের প্রার্থী আজমত উল্লা খান, টেবিলঘড়ি প্রতীকে জায়েদা খাতুন (সাবেক মেয়র জাহাঙ্গীর আলমের মা), লাঙ্গল প্রতীকে জাতীয় পার্টির প্রার্থী এম এম নিয়াজ উদ্দিন, হাতপাখা প্রতীকে ইসলামী আন্দোলন বাংলাদেশের গাজী আতাউর রহমান, গোলাপ ফুল প্রতীকে জাকের পার্টির মো. রাজু আহাম্মেদ, মাছ প্রতীকে গণফ্রন্টের প্রার্থী আতিকুল ইসলাম। এ ছাড়া স্বতন্ত্র থেকে মেয়র পদে ঘোড়া প্রতীকে মো. হারুন-অর-রশীদ ও হাতি প্রতীকে সরকার শাহনূর ইসলাম প্রতিদ্বন্দ্বিতা করেছেন।
\end{document}